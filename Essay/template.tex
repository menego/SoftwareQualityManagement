
\documentclass[a4paper, 10pt, conference]{ieeeconf}
\overrideIEEEmargins

% The following packages can be found on http:\\www.ctan.org
%\usepackage{graphics} % for pdf, bitmapped graphics files
%\usepackage{epsfig} % for postscript graphics files
%\usepackage{mathptmx} % assumes new font selection scheme installed
%\usepackage{times} % assumes new font selection scheme installed
%\usepackage{amsmath} % assumes amsmath package installed
%\usepackage{amssymb}  % assumes amsmath package installed


% *** GRAPHICS RELATED PACKAGES ***
%
\usepackage{graphicx}

% *** PDF, URL AND HYPERLINK PACKAGES ***
%
\usepackage{hyperref}

% correct bad hyphenation here
\hyphenation{op-tical net-works semi-conduc-tor}

% Load package for Checkmark symbol:
\usepackage{bbding}

\title{Topic 2 -- ``Developing successful cloud software'' \\
\large{Alternative title}} % subtitle that should be changed

\author{Nicola Meneghetti \\
	Email: menego1983@gmail.com \\
	Course Id: 876a5df21ca
}

\begin{document}

\maketitle
\thispagestyle{empty}
\pagestyle{empty}

\begin{abstract}

This is the abstract of the template document. It serves only as a starting point for editing. The title is an example with Topic 1. Similarly, you can choose Topic 2 -- ``..'' or 3 -- ``..''. Additionally, you can add an alternative title. It can be a telling or even funny reference to your particular context.

\end{abstract}

\section{Introduction}

During the last decade we experienced a big change in the way software is developed and distributed. In this regard, cloud computing plays a crucial part where all the rules of classical software development are evolving to address the ubiquity and the faster time to market demands. An example of new architecture that leverages cloud computing is given by micro services, in which applications are essentially a composition of independent, uncoupled entities that communicate between each other. New roles emerged, such as DevOps, where development and operation converge into a single figure with the ability to implement and coordinate the high level of automation required by this new approach. Consolidated techniques like object oriented programming are shifting to a more abstracted paradigm such as MDE (Model Driven Engineering) where models are the building blocks of software systems, separating design and logic from implementation details~\cite{overview-platforms, fuggetta2014software}. All these aspects are still growing towards maturity and they bring a whole new set of challenges and considerations.

Furthermore many efforts are being made in the creation of an ecosystem able to assist the developers in their work. As reported by G. Fylaktopoulos et al.~\cite{overview-platforms}, we already have at our disposal a whole set of integrated development environments (IDE) that try to offer a comprehensive set of functionalities to create, test and deploy cloud software. *give examples and stress the lack of important aspects covered*

paragraph 3 evolution in abstraction MDE SOA ASDaaS
Furthermore 

paragraph 4 lack of security in all aspect of research in cloud
However...

\section{Ideas}

Following the considerations exposed in the introduction, I formulated 3 research questions:
\begin{enumerate}
 	\item In this new landscape characterized by the convergence towards 'everything as a service' and the increasing abstraction of processes~\cite{automatic-dev}, how is software development impacted? What is the main trend in this regard?
 	\item In light of all the considerations that we made about the requirements of cloud services development. What are the changes in the software development tool-chain that the already existing solutions do not offer yet? 
	\item How can we improve and standardize security methodologies for cloud software development?
\end{enumerate}



% References ~\cite{id}, a-word\footnote{Lorem ipsum dolor sit amet, consectetur adipiscing elit.}
\bibliographystyle{IEEEtran}
% your references should be in the file ref.bib
\bibliography{IEEEabrv,ref}


\end{document}
