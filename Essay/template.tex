
\documentclass[a4paper, 10pt, conference]{ieeeconf}
\overrideIEEEmargins

% The following packages can be found on http:\\www.ctan.org
%\usepackage{graphics} % for pdf, bitmapped graphics files
%\usepackage{epsfig} % for postscript graphics files
%\usepackage{mathptmx} % assumes new font selection scheme installed
%\usepackage{times} % assumes new font selection scheme installed
%\usepackage{amsmath} % assumes amsmath package installed
%\usepackage{amssymb}  % assumes amsmath package installed


% *** GRAPHICS RELATED PACKAGES ***
%
\usepackage{graphicx}

% *** PDF, URL AND HYPERLINK PACKAGES ***
%
\usepackage{hyperref}

% correct bad hyphenation here
\hyphenation{op-tical net-works semi-conduc-tor}

% Load package for Checkmark symbol:
\usepackage{bbding}

\title{Topic 2 -- ``Developing successful cloud software'' \\
\large{Alternative title}} % subtitle that should be changed

\author{Nicola Meneghetti \\
	Email: menego1983@gmail.com \\
	Course Id: 876a5df21ca
}

\begin{document}

\maketitle
\thispagestyle{empty}
\pagestyle{empty}

\begin{abstract}

This is the abstract of the template document. It serves only as a starting point for editing. The title is an example with Topic 1. Similarly, you can choose Topic 2 -- ``..'' or 3 -- ``..''. Additionally, you can add an alternative title. It can be a telling or even funny reference to your particular context.

Lorem ipsum dolor sit amet, consectetur adipiscing elit. Donec elit lorem, vulputate ut fermentum tincidunt, pulvinar et lorem. Phasellus lacinia condimentum tellus, sed iaculis tortor ultrices ac. Nunc iaculis et massa eget viverra. Vestibulum laoreet euismod risus. Curabitur bibendum augue eu risus vulputate fermentum.

\end{abstract}

\section{Introduction}

In the last decade we experienced a big change in the way software is developed and distributed. In this regard cloud computing plays a crucial part where all the rules of classical software development are being changed to address the ubiquity and the faster time to market demands. An example of new architecture that leverages cloud computing is given by microservices, where new applications are essentially a composition of independent, uncoupled entities that communicate between each other. New roles emerged, such as DevOps, where development and operation converge into a single figure with the ability to implment the high level of automation required by this new approach. 




\section{Ideas}

Aliquam euismod~\cite{fuggetta2014software} orci accumsan facilisis porta. Donec efficitur, dui pharetra faucibus dictum, neque lacus mollis felis, in vulputate lorem arcu in ipsum. Suspendisse bibendum ac mauris in tincidunt. Nam condimentum ultricies ipsum, id vulputate orci rhoncus quis. Donec eget metus felis. Phasellus id ex aliquet, vulputate orci ut, eleifend risus. Donec quis vestibulum erat, non dictum mi. Pellentesque ac egestas sem. In a lacinia massa. Ut metus tellus, maximus nec aliquam eu, blandit sit amet eros. Morbi lacinia velit ipsum, vel molestie lorem molestie eu. Nam vel nisi ante. Nulla tortor nisl, vestibulum eu dui sit amet, convallis mollis felis. Suspendisse vel aliquet orci.

% References ~\cite{id}, a-word\footnote{Lorem ipsum dolor sit amet, consectetur adipiscing elit.}
\bibliographystyle{IEEEtran}
% your references should be in the file ref.bib
\bibliography{IEEEabrv,ref}


\end{document}
